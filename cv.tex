\documentclass[10pt,a4paper,sans]{moderncv}        % possible options include font size ('10pt', '11pt' and '12pt'), paper size ('a4paper', 'letterpaper', 'a5paper', 'legalpaper', 'executivepaper' and 'landscape') and font family ('sans' and 'roman')

% moderncv themes
\moderncvstyle{classic}                             % style options are 'casual' (default), 'classic', 'banking', 'oldstyle' and 'fancy'
\moderncvcolor{blue}                               % color options 'black', 'blue' (default), 'burgundy', 'green', 'grey', 'orange', 'purple' and 'red'
%\renewcommand{\familydefault}{\sfdefault}         % to set the default font; use '\sfdefault' for the default sans serif font, '\rmdefault' for the default roman one, or any tex font name
%\nopagenumbers{}                                  % uncomment to suppress automatic page numbering for CVs longer than one page

% character encoding
\usepackage[utf8]{inputenc}                       % if you are not using xelatex ou lualatex, replace by the encoding you are using
%\usepackage{CJKutf8}                              % if you need to use CJK to typeset your resume in Chinese, Japanese or Korean

\usepackage{url}
\usepackage{hyperref}

% adjust the page margins
\usepackage[scale=0.95]{geometry}
%\setlength{\hintscolumnwidth}{3cm}                % if you want to change the width of the column with the dates
%\setlength{\makecvtitlenamewidth}{10cm}           % for the 'classic' style, if you want to force the width allocated to your name and avoid line breaks. be careful though, the length is normally calculated to avoid any overlap with your personal info; use this at your own typographical risks...

% personal data
\name{}{Yash Khare}
\address{Kollam, Kerala}{India}% optional, remove / comment the line if not wanted; the "postcode city" and "country" arguments can be omitted or provided empty
\phone[mobile]{+91 6397812260}                   % optional, remove / comment the line if not wanted; the optional "type" of the phone can be "mobile" (default), "fixed" or "fax"
\email{yashsja@gmail.com}                               % optional, remove / comment the line if not wanted
\homepage{yashk2000.github.io}                         % optional, remove / comment the line if not wanted
\social[linkedin]{yashk2000}                        % optional, remove / comment the line if not wanted
\social[twitter]{\_p0lar\_bear}                             % optional, remove / comment the line if not wanted
\social[github]{yashk2000}                              % optional, remove / comment the line if not wanted

% bibliography adjustements (only useful if you make citations in your resume, or print a list of publications using BibTeX)
%   to show numerical labels in the bibliography (default is to show no labels)
\makeatletter\renewcommand*{\bibliographyitemlabel}{\@biblabel{\arabic{enumiv}}}\makeatother
%   to redefine the bibliography heading string ("Publications")
%\renewcommand{\refname}{Articles}

% bibliography with mutiple entries
%\usepackage{multibib}
%\newcites{book,misc}{{Books},{Others}}
%----------------------------------------------------------------------------------
%            content
%----------------------------------------------------------------------------------
\begin{document}
%\begin{CJK*}{UTF8}{gbsn}                          % to typeset your resume in Chinese using CJK
%-----       resume       ---------------------------------------------------------
\makecvtitle

\section{OBJECTIVE}
\cvitem{}{To gain exposure, enhance my skills and always be ready to learn new things.}

\section{EDUCATION}
\cventry{2018-2022 Ongoing}{B.Tech in Computer Science and Engineering}{Amrita Vishwa Vidyapeetham, Amritapuri}{}{}
{\textit{CGPA: 9.18/10}}{}  % arguments 3 to 6 can be left empty
\cventry{August 2020}{AI Singapore Summer School}{}{}{}
{I was selected for the AI Summer School 2020 hosted by AI Singapore. Over the course, I learned about Reinforcement Learning, Federation Learning, AI in education and healthcare and much more on how to pursue a research oriented career in this field.}
\cventry{July 2019}{Undergraduate Summer School, Indian Institute of Science}{}{}{}
{This program is a course for introduction into the fields of research where students are most involved currently. It is mainly meant for final and pre-final year students, but sophomores can also apply. I was selected for the program in my sophomore year itself after clearing the application and interview phases  being the only sophomore accepted out of the 90 selected students all over India.}{}
\cventry{2018}{St.Joseph's Academy, Higher Secondary Education ISC}{}{}{}
{\textit{Percentage: 94.25\%  }}{}  % arguments 3 to 6 can be left empty
\cventry{2016}{St.Joseph's Academy, Higher Education ICSE}{}{}{}
{\textit{Percentage: 92.2\%  }}{}  % arguments 3 to 6 can be left empty

\section{EXPERIENCE}
\cventry{September 2020 - Present}{MLH Fellow}{}{}{}
{I was selected as an MLH(Major League Hacking) Fellow, one of around 170 students selected for my batch out of 20000 applicants. I will be working on making new projects and experimenting with new technologies by collaborating in small groups on a series of short hackathon sprints.}
\cventry{May 2020 - September 2020}{Google Summer of Code}{}{}{}
{My proposal, "Computer Vision Based PPI Tool Version 2.0", under the Mifos Initiative was accepted for GSoC 2020. Over the summer I will work on training models to accurately detect and classify objects in household environments and build an Android app to leverage MLKit for using tflite models and automatically fill PPI surveys. My work also involved collecting data of the needed objects, performing augmentation to increase the dataset size and training and converting models to tflite on the gathered data. }
\cventry{May 2020 - July 2020}{Intern at Instruments Research and Development Establishment(IRDE)}{}{}{}
{IRDE(a DRDO establishment), was developing a fever screening system. The system uses a normal camera to capture video and an IR Camera to detect temperature. I worked on developing the software and integrating it with the hardware. My work involved detecting faces in the RGB video and scale these inputs to match the scale of the IR camera such that temperature of only the facial regions could be extracted for which I used deep learning. A few parameters also change as the temperature of the IR camera changes when it is in use. I developed machine learning algorithms to automatically adjust the parameters so as to give the correct output. I also developed a GUI in python.}
\cventry{August 2019 - Present}{GitHub Campus Expert}{}{}{}
{As GitHub Campus Expert, I receive training and mentorship from GitHub employees and support to help in the growth the developer community on my campus}
\cventry{November 2019 - December 2019}{Intern at Defence Research and Development Organization(DRDO)}{}{}{}
{I interned at IRDE, a DRDO establishment. During the internship, my work involved digital image processing, computer vision and automatic target detection using background differencing, frame differencing, and difference fusion. An algorithm was also developed by me for automatic detection of moving ground targets, viz. vehicle, human, etc. in image sequences captured by an infrared (thermal) imaging system. Experimental results demonstrated that the proposed algorithm can detect intruding targets in infrared imaging video with good accuracy}
\cventry{November 2019 - January 2020}{Google Code-In Mentor}{}{}{}
{Google Code-in is a contest to introduce pre-university students (ages 13-17) to open source software development. I have been invited by the Wikimedia Foundation and FOSSASIA as a mentor for the Google Code-In 2019.}
\cventry{May 2019 - August 2019}{FOSSASIA Internship}{}{}{}
{I got selected as a FOSSASIA intern in  May 2019. I overhauled cloud deployment of 2 applications, resulting in reduced run time performance by 30\%. I helped in developing the hardware simulation, Badge Magic Android, of a LED name badge, by passing the 2D array into a filter of animation specific algorithm; this enabled people without the hardware to experience the hardware beforehand. My work was also on the Phimp.me Android application which is photo editing tool using OpenCV. For both of these apps, I automated PlayStore and F-droid deployment process and improved the build time by 5 minutes using Fastlane tool, bash scripting, and continuous integration.}{}  % arguments 3 to 6 can be left empty
\cventry{July 2018 - Present}{Member and mentor at amFOSS}{}{}{}
{amFOSS is the Free and Open Source Software club  of my college. I am an active member and also mentor my juniors and get them exposed to new technologies and open source as well.}

\section{ACHIEVEMENTS}
\cvlistitem{Finished as a \textbf{runner up in the IEEE GovTechThon'20}, out of the 100+ teams selected for the hackathon.}
\cvlistitem{Got selected for \textbf{HackMIT 2020}, which is the annual hackathon organized by Massachusetts Institute of Technology.}
\cvlistitem{My team was among the \textbf{top teams out of 200+ teams} that participated in Hac'kp 2020, the international hackathon organized by the Kerala Police Cyberdome.}
\cvlistitem{Got invited to FOSSASIA OpenTech Summit 2020, Singapore, to give a talk on \textbf{The Optimal Pathway to Deep Learning}}
\cvlistitem{Top contributor to Kiwix Android, a Wikimedia offliner, with \textbf{100+} contributions made.}
\cvlistitem{My paper on \textbf{Infrared Image Enhancement using Convolution Matrices} got selected to be presented in the \textbf{International Conference on Optics and Electro-Optics 2019(ICOL 2019)} held at \textbf{Instruments Research and Development Establishment}, a premier \textbf{DRDO} establishment working in the field of Electro-optics}
\cvlistitem{Got selected for \textbf{Hack The North 2019}, held at the University of Waterloo, Canada(travel funding provided)}
\cvlistitem{Won \textbf{2nd prize in IBM-Cloud Category} in FOSSASIA UNESCO Hackathon held in Singapore.}
\cvlistitem{FOSSASIA \textbf{OpenTech Night winner}: Got invited to \textbf{FOSSASIA Open Tech Summit, Singapore} in March 2019.}
\cvlistitem{Finished among the \textbf{top 3 participants in Kharagpur Winter of Code(KWoC) 2019}.}

\section{PROJECTS}
\cventry{Helping Hands September 2020}{Tech stack: Python, Jupyter Notebook, Flask, Flutter, Microsoft Cognitive APIs}{}{}{}
{Helping Hands aims to bridge the gap between them and the visual world by leveraging the power of Deep Learning which can be made accessible even on low-ended devices with a lucid User-Interface that would exactly allow them to better understand the world around. This project also won the first prize in the first sprint of the MLH Fellowship. Link to project:{\newline}
\url{https://github.com/HarshCasper/HelpingHand}}{}
\cventry{Tweegenous March 2019}{Tech stack: Python, Jupyter Notebook}{}{}{}
{This project is used to collect tweets from twitter in different languages using NLP. It was developed as a part of the FOSSASIA-UNESCO Hackathon, in Singapore, in which my team won the 2nd place in IBM-Cloud Category. The tool was designed for people who speak indigenous languages. It collects tweets related to natural disaster and translates them in the language desired by the user and alerts people instantly if there is a natural calamity or any disaster headed their way by translating tweets. It is a two way system, for both the authorities and people.  Link to project:{\newline}
\url{https://github.com/tweegenous}}{}
\cventry{Vision PPI March 2020 - Present}{Tech stack: Python, Jupyter Notebook, Tensorflow, Kotlin, Java, MLKit, Retrofit, RXJava}{}{}{}
{I worked on Vision PPI as a part of Google Summer of Code'20. I trained models for image labelling using Tensorflow and converted them into a TensorflowLite model to be deployed on an Android app. The Android app is built using Java and Kotlin to perform object detection and automatically fill PPI Surveys based on the objects detected in an image. Link to project:{\newline}
\url{https://github.com/openMF/ppi-vision}}{}
\cventry{Psychic-CCTV: August 2020 - Present}{Tech stack: Python, Jupyter Notebook, PyTorch, PyQt}{}{}{}
{Psychic-CCTV is a video analysis tool capable of analysing CCTV footage, or any low quality video. The tool performs super-resolution to enhance the video quality, object detection to obtain all objects of interest and sound extraction to separate different sounds by their sources for better analysis. Link to project:{\newline}
\url{https://github.com/Fireboltz/Psychic-CCTV}}{}
\cventry{Ocellus: July 2020 - August 2020}{Tech stack: React, Typescript, python}{}{}{}
{This project does OSINT data analysis: IP address scans, IP address heatmaping, tracking IPs, tracking mac addresses of a system, phone number and email verification, and blacklist and domain analysis. This was developed as a part of Hac'KP, an international hackathon organized by the Kerala Police Cyberdome. The project was ranked under the top 20 out of 200+ participants.}
\cventry{Kiwix Android January 2020 - July 2020}{Tech Stack: Kotlin, Java, Android, RxJava, Dagger}{}{}{}
{Kiwix is an offline reader for Web content. One of its main purposes is to make Wikipedia available offline. This is done by reading the content of a file in the ZIM format, a highly compressed open format with additional meta-data. I am one of the top contributors to this project. Link to project:{\newline}
\url{https://github.com/kiwix/kiwix-android}}{}
\cventry{amFOSS CMS April 2020 - Present}{Tech Stack: Flutter, Dart, GraphQL}{}{}{}
{This is a flutter application for the amFOSS CMS, compatible with android and iOS operating systems. Using the application, club members can login into the Club Management System and view club attendance, their profile and status updates(daily emails) for which data is fetched using the CMS APIs which were also made by amFOSS members itself. Features like notifications for club meetings, event, a todo for members to kep track of ,etc are also present. Link to project:{\newline}
\url{https://gitlab.com/amfoss/cms-mobile}}{}
\cventry{Phimpme November 2018 - January 2020}{Tech stack: Java, XML, Android, OpenCV}{}{}{}
{Phimpme is an open source photo editing application designed for android phones. I am one of the top contributors in this project and have fixed several bugs and made several new features. I am also one of the maintainers of this project. Link to project:{\newline} \url{https://github.com/fossasia/phimpme-android}}{}  % arguments 3 to 6 can be left empty
\cventry{Badge-Magic January 2019 - January 2020}{Tech Stack: Kotlin, XML, Android}{}{}{}
{Badge-Magic is an android application which is used to control LED Badges. I have been a core contributor to the project and have helped build this project from scratch. I also help maintain this project. Link to project:{\newline} \url{https://github.com/fossasia/badge-magic-android}}{}  % arguments 3 to 6 can be left empty
\cventry{Asha-SOS July 2019}{Tech Stack: JavaScript, HTML, CSS, Bootstrap}{}{}{}
{This is a project for disaster management in case of floods when due to loss of internet connection, people are not able to send for help. Our project helps in providing a network in case of floods. A device called a Node-MCU is used to provide an wifi network. Link to project:{\newline} \url{https://github.com/kochi-hackathon/AshaSOS}}{}  % arguments 3 to 6 can be left empty
\cventry{Temple App June 2019 - June 2020}{Tech stack: Java, XML, Android, Firebase}{}{}{}
{An Android app which handles the information about a temple. People can register and keep a track of all poojas, donations made to the temple. I am one of the core developers and maintainers of this project. Link to project:{\newline} \url{https://github.com/amfoss/TempleApp}}{}  % arguments 3 to 6 can be left empty

\section{VOLUNTEERING}
\cvlistitem{\textbf{MLH Hackcon 2020:} MLH Hackcon is a conference where hackathon organizers and hacker community leaders can come together. I was one of the team members who helped in handling the virtual GitHub booth with the GitHub Education Team.}
\cvlistitem{\textbf{Amrita inCTF 2019:} Amrita inCTF is India's biggest Capture the Flag contest held by team bi0s, India's top cyber security team, at my college. I was responsible for event management and organization for the event.}
\cvlistitem{\textbf{MLH Local Hack Day 2019:} This is a series of 3 events spanning across one academic  year of college. I have successfully organized Learn and Build events which were very well received.}
\cvlistitem{\textbf{Hacktoberfest Meetup Amritapuri:} This was a 2 day workshop to introduce beginners to Open Source via means of Hacktoberfest(a program by Digital Ocean). I helped organize this event on 9-10 October, 2019 and took sessions for the attendees to get them started with Open Source Contributions.}
\cvlistitem{\textbf{Programming Essentials Workshop 2019:} This is a 6 week long workshop(starting in August) to introduce freshers to basics of programming in languages such as C and Python. Juniors are also exposed to the world of Open Source Software and are taught about Git and GitHub. I helped in taking sessions and mentoring over 60 students as a part of this workshop.}
\cvlistitem{\textbf{CIR Road to Excellence 2019:} This workshop is help by the placement cell of my college, CIR. There were several tracks in this workshop, out of which I was responsible for organizing and taking sessions in the Android Development Track. I took a workshop for a batch of 60-70 students on developing Android apps using Java and Kotlin.}

\section{LANGUAGES}
\cvitemwithcomment{\textbf{English}}{Full Professional Proficiency}{}
\cvitemwithcomment{\textbf{Hindi}}{Native Tongue, Full Professional Proficiency}{}

\section{COMPUTER SKILLS}
\cvitem{\textbf{OS}}{Linux(Debian and Fedora), Windows}
\cvitem{\textbf{Languages}}{Python, Java, Kotlin, XML, C, C++, Bash}
\cvitem{\textbf{VCS}}{Git, Mercurial}
\cvitem{\textbf{Other Skills}}{Android Development, Machine Learning, Computer Vision, OpenCV, Problem Solving}

\section{INTERESTS}
\cvitem{\textbf{Technical}}{Computer Vision, Deep Learning}
\cvitem{\textbf{Hobbies}}{Reading, Travelling, Singing, Guitarist, Contributing to Open Source}

\section{PERSONAL DETAILS}
\cvitem{\textbf{DOB}}{12 November, 2000}
\cvitem{\textbf{Residence}}{Kollam, Kerala, India}
\cvitem{\textbf{Status}}{Student}

\clearpage
%\clearpage\end{CJK*}                              % if you are typesetting your resume in Chinese using CJK; the \clearpage is required for fancyhdr to work correctly with CJK, though it kills the page numbering by making \lastpage undefined
\end{document}


%% end of file `template.tex'.
